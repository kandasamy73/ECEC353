\documentclass[10pt]{article}

\usepackage[letterpaper, left=1in, right=1in, top=1in, bottom=1in]{geometry}
\usepackage{minted, color, setspace}
\usepackage{titlesec}
\usepackage{graphicx, subfigure}
\usepackage{hyperref, url}

\hypersetup{%
    pdfsubject={LaTeX Document},
    colorlinks=true,   % false: boxed links; true: colored links
    linkcolor=blue,    % color of internal links
    citecolor=blue,    % color of links to bibliography
    urlcolor=blue,     % color of external links
    unicode=false      % non-Latin characters in Acrobat's bookmarks
}

\makeatletter
\def\maketitle{\begin{center}{\bfseries \Large ECE-C353 Systems Programming \\ \@title} \end{center}\vspace{10pt}}
\makeatother

\titleformat{\section}{\normalfont\bfseries}{}{5pt}{}
\titlespacing{\section}{-5pt}{12pt plus 4pt minus 2pt}{0pt plus 2pt minus 2pt}

\setlength{\parindent}{0pt}
\setlength{\parskip}{8pt}

\definecolor{lightgray}{rgb}{0.95,0.95,0.95}

\title{Project 1 -- Writing a Basic Shell}

\begin{document}
\maketitle

\section{The Big Idea}

In this assignment you will be developing a user shell (i.e. command
line interface) similar to \texttt{bash}.  Upon completion, your shell
must be able to perform the following operations:

\begin{itemize}
    \item Display the current working directory within the prompt
        (before the dollar sign).

\begin{minted}[bgcolor=lightgray]{text}
/home/jshack/pssh$
\end{minted}

    \item Run a single command with optional input and output
        redirection.  Command line arguments must be supported.

\begin{minted}[bgcolor=lightgray]{bash}
/home/jshack/pssh$ ./my_prog arg1 arg2 arg3
/home/jshack/pssh$ ls -l > directory_contents.txt
/home/jshack/pssh$ grep -n "test" < input.txt > output.txt
\end{minted}

    \item Run multiple pipelined commands with optional input and output
        redirection. Naturally, command line arguments to programs must
        still be supported.

\begin{minted}[bgcolor=lightgray]{bash}
/home/jshack/pssh$ ./my_prog arg1 arg2 arg3 | wc -l > out.txt
/home/jshack/pssh$ ls -lh | awk '{print $9 " is " $5}'
/home/jshack/pssh$ ps aux | grep bash | grep -v grep | awk '{print $2}'
\end{minted}

    \item Implement the builtin command `\texttt{exit}', which will
        terminate the shell.

\begin{minted}[bgcolor=lightgray]{bash}
/home/jshack/pssh$ exit
\end{minted}

    \item Implement the builtin command `\texttt{which}'. This command
        accepts 1 parameter (a program name), searches the system
        \texttt{PATH} for the program, and prints its full path to
        \texttt{stdout} if found (or simply nothing if it is not found).
        If a fully qualified path or relative path is supplied to an
        executable program, then that path should simply be printed to
        \texttt{stdout}.  If the supplied program name is another
        builtin command, your shell should indicate that in a message
        printed to \texttt{stdout}. The behavior should be identical to
        \texttt{bash}'s builtin \texttt{which} command.

\begin{minted}[bgcolor=lightgray]{text}
/home/jshack/pssh$ which ls
/bin/ls
/home/jshack/pssh$ which lakjasdlkfjasdlkfj
/home/jshack/pssh$ which exit
exit: shell built-in command
/home/jshack/pssh$ which which
which: shell built-in command
/home/jshack/pssh$ which man
/usr/bin/man
\end{minted}

\end{itemize}


\section{The Parser}

Writing a command line shell is obviously going to require that you be
able to parse the text the user types into the prompt into a more
meaningful structure that is easier for you (the developer) to handle.
Since text processing is not the major focus of this course (hint: it's
actually systems programming), I have provided you with a parser as well
as a basic skeleton for your shell program.  I call this shell program
skeleton the \textbf{Pretty Simple SHell (pssh)}.  The main input loop
looks like this:

\begin{minted}[bgcolor=lightgray]{c}
...
#define DEBUG_PARSE 0
...
    while (1) {
        cmdline = readline (build_prompt());
        if (!cmdline)       /* EOF (ex: ctrl-d) */
            exit (EXIT_SUCCESS);

        P = parse_cmdline (cmdline);
        if (!P)
            goto next;

        if (P->invalid_syntax) {
            printf ("pssh: invalid syntax\n");
            goto next;
        }

#if DEBUG_PARSE
        parse_debug (P);
#endif
        execute_tasks (P);

    next:
        parse_destroy (&P);
        free(cmdline);
    }
...
\end{minted}

As you can see, our command line shell reads input from the user one
line at a time via \texttt{readline()}, which returns a \texttt{char*}
pointing to memory allocated on the heap containing a \texttt{NULL}
terminated string of what the user typed in before hitting Enter.  This
is passed to the provided \texttt{parse\_cmdline()} API function I have
provided to you.  The implementation can be found in \texttt{parse.c}
and the API function declarations can be found in \texttt{parse.h}.  The
\texttt{parse\_cmdline()} API returns a \texttt{Parse*}, which points to
heap memory.  The \texttt{Parse} structure is defined in
\texttt{parse.h} as follows:

\begin{minted}[bgcolor=lightgray]{c}
typedef struct {
    Task* tasks;         /* ordered list of tasks to pipe */
    int   ntasks;        /* # of tasks in the parse */

    char* infile;        /* filename of 'infile'  */
    char* outfile;       /* filename of 'outfile' */

    int background;      /* run process in background? */
    int invalid_syntax;  /* parse failed */
} Parse;
\end{minted}

as is the \texttt{Task} structure:

\begin{minted}[bgcolor=lightgray]{c}
typedef struct {
    char* cmd;
    char** argv;   /* NULL terminated array of strings */
} Task;
\end{minted}

As you can see, the \texttt{Parse} data structure contains all of the
parsed information from a command line with following anatomy:

\begin{minted}[bgcolor=lightgray]{text}
command_1 [< infile] [| command_n]* [> outfile] [&]
\end{minted}

where:

\begin{itemize}
    \item Items in brackets [ ] are optional
    \item Items in starred brackets [ ]* are optional but can be repeated
    \item Non-bracketed items are required
\end{itemize}

In other words, I have done all the ``hard work'' for you.  Your job
will simply to be to take a \texttt{Parse} structure and implement all
of the process creation and management logic generally provided by a
shell.  In order to make it obvious how to work with a \texttt{Parse}
structure, I have provided the \texttt{parse\_debug()} API function,
which simple prints the contents of a \texttt{Parse} to the screen.
Your job, for this project, largely involves writing logic for the
provided \texttt{execute\_tasks()} function in \texttt{pssh.c}.  I
recommend you start by simply getting the execution of single commands
working.


\section{Executing Single Commands}

Naturally, this is a simple \texttt{vfork()} and \texttt{exec()}
problem.  Get this working first.  Also, remember that \texttt{exec()}
is not a real function, but is rather a term used to refer to the family
of \texttt{exec} functions: \texttt{execl(), execlp(), execle(),
execv(), execvp(), execvpe()}.  Read the manual page for exec to figure
out which one you want to use!

\begin{minted}[bgcolor=lightgray]{bash}
~$ man exec
\end{minted}

Once your \texttt{pssh} is capable of executing simple commands with
arguments, such as:

\begin{minted}[bgcolor=lightgray]{bash}
$ ls -l
\end{minted}

you need to add support for input and output redirection using the
\texttt{<} and \texttt{>} command line operators... just like in
\texttt{bash}!  This is actually easier than in seems.  Hint: you will
need to modify the file descriptors tables.  Once you have this working,
you will be able to do great things like:

\begin{minted}[bgcolor=lightgray]{bash}
$ ls -l > directory.txt
$ grep "some phrase" < somefile.txt
\end{minted}

Great job!  Next step is connecting multiple commands together with
pipes!


\section{Executing Multiple Pipelined Commands}

This is actually very similar to what we did in lecture using the
\texttt{dup2()} system call.  In fact, it is nearly exactly the
same--just far more practical.  Not much needs to be said here other
than that, unlike our example in lecture, you must be able to pipeline
more than just two processes -- you must pipeline all of the commands in
the \texttt{tasks} array found in the \texttt{Parse} structure.


\section{Adding the Built-in Commands}

Since it simply ends the \texttt{pssh} process, the built-in command
\texttt{exit} is going to be the easiest.  Implement that first.  Now,
the \texttt{which} command is going to be a little bit more complicated.
I recommend looking at the \texttt{command\_found()} function in
\texttt{pssh.c} and use that as a working base.  Also, be sure to read
the manual page for the \texttt{access()} system call:

\begin{minted}[bgcolor=lightgray]{text}
~$ man access
\end{minted}

Keep in mind that, with the exception of \texttt{exit}, any built-in
command should support input and output redirection.  For example, the
following should work:

\begin{minted}[bgcolor=lightgray]{text}
$ which man > man_location.txt
$ cat man_location.txt
/usr/bin/man
\end{minted}

This may sound more challenging than it actually is.  Did you figure it
out?

\section{Building the Code}

The provided code uses a \texttt{Makefile} to help build the code.  All
you need to do to build the program is run the \texttt{make} command
within the source directory:

\begin{minted}[bgcolor=lightgray]{text}
~/src/pssh$ ls
builtin.c  Makefile  parse.h
builtin.h  parse.c   pssh.c
~/src/pssh$ make
gcc -g -Wall -c builtin.c -o builtin.o
gcc -g -Wall -c parse.c -o parse.o
gcc -g -Wall -c pssh.c -o pssh.o
gcc  builtin.o  parse.o  pssh.o -Wall -lreadline -o pssh
~/src/pssh$ ls
builtin.c  Makefile  parse.o  pssh.o
builtin.h  parse.c   pssh
builtin.o  parse.h   pssh.c
\end{minted}

and just like that, all of the necessary steps to produce the
\texttt{pssh} executable will be automatically ran.  If you want to
delete all of the intermediate object files and the \texttt{pssh}
executable, simply run \texttt{make clean} within the source directory:

\begin{minted}[bgcolor=lightgray]{text}
~/src/pssh$ ls
builtin.c  Makefile  parse.o  pssh.o
builtin.h  parse.c   pssh
builtin.o  parse.h   pssh.c
~/src/pssh$ make clean
rm -f *.o
rm -f pssh
~/src/pssh$ ls
builtin.c  Makefile  parse.h
builtin.h  parse.c   pssh.c
\end{minted}

\pagebreak

\section{Submitting Your Project}

Once you have implemented all of the required features described in this
document submit your code by doing the following:

\begin{itemize}
    \item Run \texttt{make clean} in your source directory.  We must be
        able to built your shell from source and we don't want your
        precompiled executables of intermediate object files.
        \textbf{If your code does not at the very least compile, you
        will receive a zero.}

    \item Zip up your code.

    \item Name your zip file \texttt{abc123\_pssh.zip}, where abc123 is
        your Drexel ID.

    \item Upload your zip file using the Blackboard Learn submission
        link found on the course website.
\end{itemize}

\textbf{Failure to follow these simple steps will result in your project
not being graded.}

Now, have fun!

\begin{center}
\vspace{12pt}
\textbf{Due in 2 Weeks: Thursday, Feb 1st \emph{before} midnight.}
\end{center}

\end{document}
